\documentclass[11pt]{article}

\usepackage{amsmath}
\usepackage{amssymb}
\usepackage{amsthm}
\usepackage[spanish]{babel}

\usepackage{geometry}
\newtheorem{theorem}{Theorem}
\newgeometry{vmargin={15mm}, hmargin={12mm,17mm}}

\title{Demostración que dado un grafo conexo no regular $G$, entonces $\chi(G) \le \Delta(G)$}
\author{Nicolás Cagliero}

\begin{document}
\maketitle
\begin{theorem}\par
\
Sea $G$ un grafo conexo no regular, entonces $\chi(G) \le \Delta (G)$\par
\

\end{theorem}

\begin{proof}
Sea $x: \delta (x) = \delta$. Corramos BFS a partir de $x$. Como $G$ es conexo, todos los vértices de $G$ van a ser alcanzados y en particular, todo vértice que fue alcanzado es porque tiene un vecino anterior que lo agregó. Ahora vamos a querer correr Greedy pero utilizando el orden inverso al que obtuvimos corriendo BFS. De este modo, tendremos a $x$ como el último vértice en colorearse y todo vértice tendrá por lo menos un vecino posterior. Veamos que pasa cuando quiero colorear un vértice $z \ne x$, Greedy lo que va a hacer es eliminar a lo sumo $\delta(z) - 1 \le \Delta - 1$ colores, luego, podrá colorear con algún color $ \in \{1 ... \Delta\}$. Al colorear $x$, va a eliminar $\delta < \Delta$ colores $\Rightarrow$ puede usar algún color $ \in \{1 ... \Delta\}$.

\end{proof}

\end{document}
