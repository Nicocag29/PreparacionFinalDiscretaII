\documentclass[11pt]{article}

\usepackage{amsmath}
\usepackage{amssymb}
\usepackage{amsthm}
\usepackage[spanish]{babel}

\usepackage{geometry}

\newgeometry{vmargin={15mm}, hmargin={12mm,17mm}}

\title{¿Cuál es la complejidad del algoritmo de Edmonds-Karp?}
\author{Nicolás Cagliero}

\begin{document}


\maketitle
\textbf{Teorema:} el algoritmo Edmonds-Karp termina siempre con complejidad $O(n \cdot m^{2}) $ \par 
\

\begin{proof}E-K constituye una sucesión de caminos aumentantes, Cada uno de ellos encontrado usando BFS, que es $O(m)$.\par

\begin{center}
	$\therefore$ Su complejidad es $O(m \cdot \#CaminosAumentante)$\par
\end{center}
\begin{center}
	$\therefore$ Solo debemos probar que, en E-K, $\#CA = O(n \cdot m)$
\end{center}
\

Definición: un lado es (o "se vuelve") crítico en un punto dado si es usado para construir el siguiente flujo en modo forward y se satura o en modo backward y se vacía.

\begin{center}
	¿Cuántas veces puede un lado volverse crítico?\par
\end{center}

Para calcular esto, veamos unas definiciones auxiliares:\par

\begin{enumerate}
\item dados vértices $x, z$ se define, dado un flujo $f$ en el network
\[   
d_f(x,z) = 
     \begin{cases}
       \text{\(0\)}, &\quad\text{si \(x=z\)}\\
       \text{\(\infty\)}, &\quad\text{si no existe \(f-CA\) desde \(x\) a \(z\)}\\
       \text{mínima longitud de un \(f-CA\) desde \(x\) a \(z\)}, &\quad\text{si existe}
     \end{cases}
\]

\item Sean $f_0, f_1, f_2, ...$  los flujos que se van obteniendo en E-K:
\begin{center}
	$d_k(x) = d_{f_k}(s, x)$\par
	$b_k(x) = d_{f_k}(x, t)$\par
\end{center}

\end{enumerate}
\

Prop:  $d_k(x) \le d_{k+1}(x)$\par
\

Prop:  $b_k(x) \le b_{k+1}(x)$\par
\

Sea $\overrightarrow{xy}$ un lado que se vuelve crítico en el paso $k$, hay $2$ casos:\par
\begin{enumerate}
	\item  \begin{center}
			se saturó\par
		 \end{center}
	\item \begin{center}
			se vació\par
		\end{center}
\end{enumerate}
\begin{itemize}
\item Caso en el cual el lado saturó:  es decir, $f_k(\overrightarrow{xy}) < c(\overrightarrow{xy})$ pero  $f_{k+1}(\overrightarrow{xy}) = c(\overrightarrow{xy})$\par
$\therefore$ para pasar de $f_k$ a $f_{k+1}$ usamos un $f_k-CA$ de la forma $s...xy...t$\par
Como estamos usando E-K, ese camino es de longitud mínima $\therefore d_k(y) = d_k(x) + 1$\par
Supongamos que $\overrightarrow{xy}$ se vuelve crítico otra vez en un paso $j>k$\par
	\begin{itemize}
		\item Se vació, i.e, usamos $\overrightarrow{xy}$  backward en el paso $j$\par
		\item Se saturó\par
	\end{itemize}
Como luego del paso $k$ ya estaba saturado, para poder volverse a saturar debemos haberle devuelto el flujo en algún paso $k < i < j$\par
En cualquier caso, $\exists i: k < i \le j$ tal que $\overrightarrow{xy}$ se usa en forma backward, es decir, usamos un $f_i-CA$ de la forma $s...\overleftarrow{yx}...t$. Como E-K, este camino es de longitud mínima $\Rightarrow d_i(x) = d_i(y) + 1$\par
Entonces $d_j(t) \ge d_i(t)$
	\begin{center}
		$d_i(t) = d_i(x) + b_i(x)$\par
		$d_i(t) = d_i(y) + 1 + b_i(x)$\par
		$d_i(t) \ge d_k(y) + 1 + b_k(x)$\par
		$d_i(t) \ge d_k(x) + 1 + 1 + b_k(x)$\par
		$d_i(t) \ge d_k(t) + 2$\par
	\end{center}
$\Rightarrow d_j(t) \ge d_k(t) + 2$ si $\overrightarrow{xy}$ se satura en el paso $k$\par

\item Caso en el cual se vació: la próxima vez que se vuelva crítico será porque se saturó, o bien, se vació de vuelta $\therefore$ antes debo haberlo llenado un poco. En cualquier caso: $\exists i: k < i \le j$ tal que $\overrightarrow{xy}$ se usa en forma forward, es decir, usamos un $f_i-CA$ de la forma $s...\overrightarrow{xy}...t$. Como E-K, este camino es de longitud mínima. Luego $d_j(t) \ge d_i(t)$\par
	\begin{center}
		$d_i(t) = d_i(y) + b_i(y)$\par
		$d_i(t) = d_i(x) + 1 + b_i(y)$\par
		$d_i(t) \ge d_k(x) + 1 + b_k(y)$\par
		$d_i(t) \ge d_k(y) + 1 + 1 + b_k(y)$\par
		$d_i(t) \ge d_k(t) + 2$\par
	\end{center}
$\Rightarrow d_j(t) \ge d_k(t) + 2$ si $\overrightarrow{xy}$ se vacía en el paso $k$\par
\
\end{itemize}

Conclusión: $d_j(t) \ge d_k(t) + 2$ en ambos casos\par
\

Es decir, cuando un lado se vuelve crítico, para que pueda volver a ser crítico, la distancia entre $s$ y $t$ debe aumentar en al menos $2$\par
\

Como la distancia entre puede variar entre $1$ y $n-1 \Rightarrow$ un lado puede volverse crítico a lo sumo $O(\frac{n}{2}) = O(n)$ veces\par
\


Resumen:\par
\begin{center}
\begin{itemize}

\item cada CA vuelve crítico al menos un lado\par
\item cada lado se vuelve crítico a los sumpo $O(n)$ veces\par
\item hay $m$ lados\par

\end{itemize}
\end{center}

$\Rightarrow$ hay $O(m \cdot n)$ caminos aumentantes.

\end{proof}  

\end{document}

