\documentclass[11pt]{article}

\usepackage{amsmath}
\usepackage{amssymb}
\usepackage{amsthm}
\usepackage[spanish]{babel}

\usepackage{geometry}
\newtheorem{theorem}{Theorem}
\newgeometry{vmargin={15mm}, hmargin={12mm,17mm}}

\title{Demostración del Teorema del Matrimonio de Konig}
\author{Nicolás Cagliero}

\begin{document}
\maketitle
\begin{theorem}\par
\
Todo grafo $G$ bipartito regular ($\chi(G) = 2$ y $\delta = \Delta$) tiene un matching perfecto.
\

\end{theorem}

\begin{proof} Dado un $W \subseteq V$ definimos \[E_w = \{wv: w \in W\}\] \par
\

Sea $X$ e $Y$ las partes de $G$\par
Sea $S \subseteq X$\par
Sea $l \in E_s \Rightarrow \exists x \in S, y \in Y: l = xy$. Por lo tanto $y \in \Gamma(x) \Rightarrow y \in \Gamma(S) \Rightarrow l \in E_{\Gamma(S)}$\par
\

Luego $E_s \subseteq E_{\Gamma(S)} \Rightarrow |E_s| \le |E_{\Gamma(S)}|$\par
\

Veamos ahora $E_w$ cuando $W \subseteq X$ o $W \subseteq Y$. Si $wv \in E_w \Rightarrow v \notin W$ pues si \[W \subseteq X \Rightarrow v \in Y\] \[W \subseteq Y \Rightarrow v \in Y\] \par
\

Luego $E_w = \{wv: v \in \Gamma(w)\}$ y esta unión es disjunta por lo recién visto. Luego \[|E_w| = \sum_{w \in W} \delta(w) = \Delta \cdot |W|\]\par
\

Aplicando todo lo aprendido, $|S| \cdot \Delta \le |\Gamma(S)| \cdot \Delta \Rightarrow |S| \le |\Gamma(S)|$. Luego, se cumple Hall y sabemos que existe un matching completo de $X$ a $Y$. Ahora solo resta ver que $|X| = |Y|$.\par
\

Sabiendo que existe matching completo de $X$ a $Y$ sabemos que $|X| \le |Y|$ pero debemos recordar que la elección de $X$ fue arbitraria, luego podemos repetir los pasos para $Y$ y llegaríamos a que existe matching completo de $Y$ a $X$. Y así llegamos a que $|X| = |Y|$.


\end{proof}

\end{document}
