\documentclass[11pt]{article}

\usepackage{amsmath}
\usepackage{amssymb}
\usepackage{amsthm}
\usepackage[spanish]{babel}

\usepackage{geometry}
\newtheorem{theorem}{Theorem}
\newgeometry{vmargin={15mm}, hmargin={12mm,17mm}}

\title{Demostración del Teorema de la cota de Hamming}
\author{Nicolás Cagliero}

\begin{document}
\maketitle
\begin{theorem}\par
\
Sea $C \subseteq \{0, 1\}^{2}$ un código binario de longitud $n$. Sea $\delta = \delta(C)$ y $t = \lfloor \frac{\delta - 1}{2} \rfloor$. Entonces \[|C| \le \frac{2^{n}}{1 + n + \binom{n}{2} + ... + \binom{n}{t}}\]
\

\end{theorem}

\begin{proof} Sea \[A = \bigcup_{v \in C} D_t(v)\] \par
\

Como $t = \lfloor \frac{\delta - 1}{2} \rfloor$, $C$ corrige $t$ errores $\Rightarrow D_t(v) \cap D_t(w) = \emptyset$ para cualquier palabra de $C$. Luego, la unión de $A$ es disjunta \[\Rightarrow |A| = \sum_{v \in C} |D_t(v)|\]\par
\

Sean $S_r(v) = \{w \in \{0, 1\}^{2} : d_H(v, w) = r \} $\par
\

\[\Rightarrow |D_t(v)| = \sum_{r = 0}^{t} |S_r(v)|\]\par
\

Ahora queda ver cuánto vale $|S_r(v)|$. Notar que si $w \in S_r(v)$, entonces $w$ difiere en exactamente $r$ bits de $v$. Tenemos una biyección entonces entre $S_r(v)$ y el conjunto de subconjuntos de $r$ bits de los $n$ bits posibles. Luego $|S_r(v)| = \binom{n}{r}$.\par
\

\[\Rightarrow |D_t(v)| = \sum_{r = 0}^{t} |S_r(v)| = \sum_{r = 0}^{t} \binom{n}{r}\]\par
\

\[\Rightarrow |A| = \sum_{v \in C} |D_t(v)| = \sum_{v \in C}(\sum_{r = 0}^{t} \binom{n}{r}) = \sum_{r = 0}^{t} \binom{n}{r} \cdot |C|\]\par
\

Por lo tanto \[|C| = \frac{|A|}{\displaystyle\sum_{r = 0}^{t} \binom{n}{r}}\]\par
\

Como $A \subseteq \{0, 1\}^{2} \Rightarrow |A| \le 2^{n}$ \[\Rightarrow |C| \le \frac{2^{n}}{\displaystyle\sum_{r = 0}^{t} \binom{n}{r}}\]

\end{proof}

\end{document}
