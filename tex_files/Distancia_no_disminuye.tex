\documentclass[11pt]{article}

\usepackage{amsmath}
\usepackage{amssymb}
\usepackage{amsthm}
\usepackage[spanish]{babel}

\usepackage{geometry}
\newtheorem{theorem}{Theorem}
\newgeometry{vmargin={15mm}, hmargin={12mm,17mm}}

\title{Demostración las distancias no disminuyen}
\author{Nicolás Cagliero}

\begin{document}
\maketitle
\begin{theorem}\par
\

\begin{enumerate}
\item Dados vértices $x, z$ se define, dado un flujo $f$ en el network
\[   
d_f(x,z) = 
     \begin{cases}
       \text{\(0\)}, &\quad\text{si \(x=z\)}\\
       \text{\(\infty\)}, &\quad\text{si no existe \(f-CA\) desde \(x\) a \(z\)}\\
       \text{mínima longitud de un \(f-CA\) desde \(x\) a \(z\)}, &\quad\text{si existe}
     \end{cases}
\]

\item Sean $f_0, f_1, f_2, ...$  los flujos que se van obteniendo en E-K:
\begin{center}
	$d_k(x) = d_{f_k}(s, x)$\par
\end{center}
\

Entonces $d_k(x) \le d_{k+1}(x)$\par
\

\end{enumerate}

\end{theorem}

\begin{proof}
Esta será una idea de la demostración y sus partes, los detalles y ciertas cuentas no serán escritas.\par
\

Sea $A = \{x : d_{k+1}(x) < d_k(x)\}$, vamos a querer que $A = \emptyset$. Supongamos que no, es decir, tomemos un $x \in  A$ y además le vamos a pedir que sea tal que $d_{k+1}(x) = min\{d_{k+1}(y) : y\in A \}$\par
\

Acá ya podemos ver que $d_{k+1}(x) < d_k(x) \le \infty \Rightarrow$ existe un $f_{k+1}-CA$ entre $s$ y $x$. Y además es obvio ver que $x \ne s$.\par
\

De esta forma, sabemos que existe un $z$ inmediatamente anterior a $x$ tal que el $f_{k+1}-CA$ es de la forma $P_{k+1} = s...zx$. Y como estamos corriendo E-K, $P_{k+1}$ es de longitud mínima.\par
\begin{center}
$\Rightarrow d_{k+1}(x) = d_{k+1}(z) + 1$
\end{center} \par 
\

De esta forma, sabemos que $\overrightarrow{zx}$ o $ \overrightarrow{xz}$.\par
\

\begin{enumerate}
\item Caso  $\overrightarrow{zx}$\par
Notar ahora que $d_{k+1}(z) < d_{k+1}(x) \Rightarrow z \notin A \Rightarrow d_k(z) \le d_{k+1}(z) < d_{k+1}(x) < \infty$. Luego, exite un $f_k -CA$ entre $s$ y $z$. Acá nos preguntamos, puedo agregar a $x$ al final de ese $f_k - CA$ dado que  $\overrightarrow{zx}$ es un lado. Si lo hacemos, dado que estamos corriendo E-K y los caminos aumentantes son de longitud mínima $\Rightarrow d_k(x) = d_k(z) + 1 \le d_{k+1}(z) + 1 = d_{k+1}(x)$. De esta forma llegamos al absurdo pues $x \in A$.\par
\

Entonces si no lo podemos agregar forward al $x$, significa que $ \overrightarrow{zx}$ está saturado en el flujo $k$, pero ya vimos que $P_{k+1}$ es un $f_{k+1}-CA \Rightarrow$ para pasar de $f_k$ a $f_{k+1}$, el lado se usó y se usó backward $\Rightarrow P'_k = s...\overleftarrow{xz} $\par
\

Como es de longitud mínima, 
\begin{align*}
d_k(z) & = d_k(x) + 1 \\
	   & > d_{k+1}(x) +1 = d_{k+1}(z) + 1 + 1 \ge d_k(z) + 2\\
	   & > d_k(z) + 2 \\
\end{align*}

$ \Rightarrow 0 > 2$. ABSURDO. \par
\

\item Caso $\overrightarrow{xz}$\par
Muy parecido al caso anterior, vamos a llegar que existe un $f_k-CA$ entre $s$ y $z$ y al querer agregar a $x$ backward llegaríamos a una contradicción, lo que nos dice que $f_k(\overrightarrow{xz}) = 0$ y como sabemos que $P_{k+1}$ es un $f_{k+1} - CA$ vamos a tener que haber usado ese lado forward de forma tal que el lado se sature un poco. De esta forma terminamos obteniendo las mismas ecuaciones que en el caso anterior y llegando al absurdo de $0>2$.\par
\
\begin{align*}
d_k(z) & = d_k(x) + 1 \\
	   & > d_{k+1}(x) +1 = d_{k+1}(z) + 1 + 1 \ge d_k(z) + 2\\
	   & > d_k(z) + 2 \\
\end{align*}



\end{enumerate}\par
\

De esta forma se llega a un absurdo de suponer que $A \ne \emptyset \Rightarrow A = \emptyset$ y el teorema queda demostrado.

\end{proof}

\end{document}
