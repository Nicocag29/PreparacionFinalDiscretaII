\documentclass[11pt]{article}

\usepackage{amsmath}
\usepackage{amssymb}
\usepackage{amsthm}
\usepackage[spanish]{babel}

\usepackage{geometry}
\newtheorem{theorem}{Theorem}
\newgeometry{vmargin={15mm}, hmargin={12mm,17mm}}

\title{Demostración complejidad Wave}
\author{Nicolás Cagliero}

\begin{document}
\maketitle
\begin{theorem}
El algoritmo Wave tiene complejidad $O(n^{3}) $
\end{theorem}

\begin{proof}
Esta será una idea de la demostración y sus partes, los detalles y ciertas cuentas no serán escritas.\par
\

Sabemos que la distancia en Networks Auxiliares sucesivos aumenta, por ende sabes que vamos a tener $O(n)$ NA en una corrida de Wave. Luego sabemos que $O(Wave) = O(n) \cdot O($compl(hallar un flujo bloqueante en un NA)$ + $compl(construir un NA))\par
\

La complejidad de armar un NA podemos ver que está dada por la cantidad de lados que puede llegar a tener $\therefore$ compl(construir un NA) $ = O(m)$\par
\

Si logramos probar que la compl(hallar un flujo bloqueante en un NA) $ = O(n^{2})$ entonces queda demostrado el teorema.\par
\

Vamos a dividir esta complejidad en varias partes, primero diferenciaremos entre olas hacia adelante y olas hacia atrás:
\begin{enumerate}
\item Olas hacia adelante: \par
\

Cuando un vértice $x$ le manda flujo a un vértice $z$, puede pasar que $\overrightarrow{xz}$ se sature o no. Definimos entonces \par
\begin{enumerate}
	\item $S = $ todos los casos donde se satura un lado.\par
	\item $P = $ todos los casos donde no se satura un lado.\par
\end{enumerate}
\

Veamos en detalle a $S$, supongamos que $\overrightarrow{xz}$ se saturó, ¿Puede volver a saturarse? Para que esto suceda, primero tiene que desaturarse $\therefore z$ debe devolverle flujo a $x \Rightarrow z$ debe estar bloqueado. Pero en las olas hacia adelante, para mandar flujo de un vértice, me fijo en sus vecinos no bloqueados, por lo tanto no puedo mandarle flujo a $z$ (bloqueado) $\Rightarrow$ no puede volverse a saturar $\overrightarrow{xz}$\par
\

A $P$ lo analizaremos más adelante.\par
\

\item Olas hacia atrás:\par
\

Cuando un vértice $x$ le devuelve flujo a un vértice $u$, puede pasar que $\overrightarrow{ux}$ quede vacío o no. Definimos entonces \par
\begin{enumerate}
	\item $V = $ todos los casos donde se vacía un lado.\par
	\item $Q = $ todos los casos donde no se vacía un lado.\par
\end{enumerate}
\

Veamos en detalle a $V$, supongamos que  $\overrightarrow{ux}$ se vació, ¿Puede volver a vaciarse? Para que esto suceda, primero tiene que llenarse, es decir, necesita que $u$ le envíe flujo a $x$ pero esto no puede pasar porque $x$ está bloqueado (le devolvió flujo a $u$).\par
\

Veamos ahora $P$ y $Q$. Al intentar balancear hacia adelante, en cada vértice vamos a lograr saturar todos los lados, salvo quizá el último, que no se satura y cuenta para la complejidad de $P$. Enviar flujo por ese lado sabemos que es $O(1) \therefore$ la complejidad de $P = O(n) \cdot$ (\#olas hacia adelante) \par
\

Similar con $Q$, al balancear hacia atrás. Podremos vaciar todos los lados, salvo quizá el último, que no se vacía y cuenta para $Q$. Luego la complejidad de $Q = O(n) \cdot $(\#olas hacia atrás). \par
\

Sabemos que \#(olas hacia atrás) = \#(olas hacia adelante). Veamos qué pasa en cada ola hacia adelante. En cada ola, salvo quizá en la última, algún vértice quedará desbalanceado y la ola lo bloquea $\therefore$ en cada ola, salvo quizá la última, se bloquea al menos un vértice y este nunca se desbloquea. $\Rightarrow O(n) = $ \#olas.\par
\

Luego, $O(S) = O(m)$,  $O(V) = O(m)$, $O(Q) = O(n^{2})$, $O(P) = O(n^{2})$ \par
\

$\therefore$ la complejidad de encontrar un flujo bloqueante es $O(m) + O(m) + O(n^{2}) + O(n^{2}) = O(m) + O(n^{2}) = O(n^{2})$


\end{enumerate}

\end{proof}

\end{document}
